\documentclass{jon-notes}

\NewDocumentCommand\Set{}{\mathbf{Set}}
\NewDocumentCommand\Psh{m}{\mathbf{P}(#1)}

\DeclareFontFamily{U}{min}{}
\DeclareFontShape{U}{min}{m}{n}{<-> dmjhira}{}
\NewDocumentCommand{\hirayo}{}{\text{\usefont{U}{min}{m}{n}\symbol{'110}}}
\NewDocumentCommand\Yo{s g}{%
  \mkern-4mu\hirayo%
  \IfValueT{#2}{%
    \IfBooleanTF{#1}{
      \parens{#2}
    }{#2}
  }%
}


\addbibresource{references/refs.bib}

\title{The equivariant uniform Kan fibration model of cubical homotopy type theory (E.~Riehl)}
\author{Notes by Jon Sterling}
\date{August 12, 2019}


\begin{document}
\maketitle

\NewDocumentCommand\Cube{}{\square}
\NewDocumentCommand\cSet{}{\Psh{\Cube}}
\NewDocumentCommand\II{}{\mathbb{I}}

\para Joint work with Awodey, Caallo, Coquand, Sattler

\para We want a $\cSet$-based model of HoTT which is a Quillen model category
equivalent to spaces (Kan complexes in simplicial sets). We may further ask
that this equivalence be a \emph{nice} functor (for instance, a triangulation),
and that $\Cube$ supports inductive constructions (in the sense of being an
Eilenberg-Zilber category).

\begin{center}
  \begin{tabular}{lll}
    \toprule
    Team & Cubes & Equivalent to spaces?\\
    \midrule
    BCH & symmetric monoidal cubes & No: consider $\II^2/\mathsf{swap}$ (Buchholtz)
    \\
    ABCFHL, A & Cartesian cubes & No: analogous argument by Sattler
    \\
    CCHM & De Morgan cubes & No: consider $\II/\mathsf{rev}$ (Buchholtz)
    \\
    CCHM & Dedekind & open problem
    \\
    \bottomrule
  \end{tabular}
\end{center}
\medskip

\para We will work with Cartesian cubes, but change the notion of fibration to
rule out the counterexample. We need $\Mor{*}{\II^2/\mathsf{swap}}$ to be a
trivial cofibration. The idea is to add an equivariance condition $j(x,y)\sigma =
j(x,y\sigma)$ relative to symmetries $\sigma$:
\[
  \begin{tikzcd}[sep = huge]
    * \arrow[r] \arrow[d] & * \arrow[r] \arrow[d] & * \arrow[r,"x"] \arrow[d] & X\arrow[d] \\
    \II^2 \arrow[r,swap,"\sigma"] \arrow[urrr,sloped,"{j(x,y\sigma)}"] & \II^2\arrow[r,->>,swap,"e"]\arrow[urr,swap,sloped,"{j(x,y)}"] & \II^2/\mathsf{swap}\arrow[r,swap,"y"] \arrow[r] & Y
  \end{tikzcd}
\]

We get the desired lift using the universal property of the quotient map $e$.



\printbibliography

\end{document}
