\documentclass{jon-notes}

\RequirePackage{favonia-symbols}
\NewDocumentCommand\SET{}{\mathbf{Set}}
\NewDocumentCommand\CAT{}{\mathbf{Cat}}
\NewDocumentCommand\CCC{}{\mathbf{CCC}}

\NewDocumentCommand\MonCCC{g}{\mathsf{T}\IfValueT{#1}{\parens{#1}}}
\NewDocumentCommand\FreeCCC{g}{\mathsf{F}\IfValueT{#1}{\parens{#1}}}
\NewDocumentCommand\ForgetCCC{g}{\mathsf{U}\IfValueT{#1}{\parens{#1}}}
\NewDocumentCommand\ThCAT{}{\mathbb{T}_{\mathbf{Cat}}}
\NewDocumentCommand\ThCCC{}{\mathbb{T}_{\mathbf{CCC}}}

\NewDocumentCommand\Atom{}{\mathsf{x}}
\NewDocumentCommand\DD{}{\mathbb{D}}
\NewDocumentCommand\CC{}{\mathbb{C}}
\NewDocumentCommand\RR{}{\mathbb{R}}
\NewDocumentCommand\DDx{}{\DD\brackets{\Atom}}
\NewDocumentCommand\CCx{}{\CC\brackets{\Atom}}
\NewDocumentCommand\RRx{}{\RR\brackets{\Atom}}
\NewDocumentCommand\ObjTerm{m}{\mathbf{1}_{#1}}
\NewDocumentCommand\Psh{m}{\mindelim{1}\widehat{#1}}

\NewDocumentCommand\Yo{s o g}{%
  \perfectedyo\IfValueT{#2}{_{#2}}%
  \IfValueT{#3}{%
    \IfBooleanTF{#1}{
      \parens{#3}
    }{#3}
  }%
}

\NewDocumentCommand\Nv{g}{\mathsf{N}\IfValueT{#1}{\parens{#1}}}


\NewDocumentCommand\Hom{omm}{\IfValueT{#1}{#1}\brackets{#2,#3}}
\NewDocumentCommand\OpCat{m}{#1^{\mathsf{op}}}


\addbibresource{references/refs.bib}

\title{Normalization for free Cartesian closed categories}
\author{Angiuli and Sterling}


\begin{document}
\maketitle

\section{Free Cartesian closed categories}

\para The theory of categories can be expressed in the language of finite
limits with classifying category $\ThCAT$; likewise, the notion of Cartesian
closure is expressed at the algebraic level by a lex functor
$\Mor{\ThCAT}{\ThCCC}$, exhibiting at the level of categories of algebras an
adjunction $\FreeCCC\dashv\ForgetCCC : \Mor{\CCC}{\CAT}$. The monad of this
adjunction $\Mor[\MonCCC]{\CAT}{\CAT}$ freely adjoins products and exponentials
to any category.

\para Let $\CCx \equiv \MonCCC\braces{\Atom}$ be the free Cartesian
closed category generated by a single object. We intend to show that there is a
normal form for morphisms in $\CCx$, and that the word problem for free
Cartesian closed categories is consequently decidable.

\section{Category of renamings}

\para The notion of normal form for an arrow in a Cartesian closed category
does not \emph{a priori} have a substitution action for arbitrary
$\CCx$-arrows. Equipping normal forms with such an action will be a
consequence of the present theorem.

However, any appropriate notion of normal form \emph{will} possess an
action for certain $\CCx$-arrows, namely the ones which arise from the
structure of finite products only. In the language of syntax, these correspond
to the substitutions which arise from structural rules (weakening, contraction,
exchange).

\para Let $\Mor|{right hook}->|[j]{\RRx}{\CCx}$ be the \emph{least wide
subcategory} of $\CCx$ which has finite products; in other words $\RRx$ has the
exact same objects as $\CCx$, but its morphisms are only the ones which arise
from product cones and their universal maps.

\section{A nerve from syntax to presheaves on renamings}

\para Let $\Psh{\CC} \equiv \Hom{\OpCat{\CC}}{\SET}$ to denote the category of
presheaves on any category $\CC$. We will write $\Mor|{right
hook}->|[\Yo]{\CC}{\Psh{\CC}}$ for the Yoneda embedding, which takes $C :\CC$
to the representable presheaf $\Hom[\CC]{-}{C}$.  Given a functor
$\Mor[f]{\CC}{\DD}$, there is a corresponding \emph{change of base} functor
$\Mor[f^*]{\Psh{\DD}}{\Psh{\CC}}$ which has both right and left adjoints, and
is therefore left exact.

\para In particular, we have a \emph{nerve} functor
$\Mor[\Nv]{\CCx}{\Psh{\RRx}}$ defined by composing the Yoneda embedding with
the change of base along $\Mor|{right hook}->|[j]{\RRx}{\CCx}$ as in the
following diagram:
\begin{equation}
  \begin{tikzpicture}[diagram,baseline = (PrR.base)]
    \node (C) {$\CCx$};
    \node (PrC) [right = of C] {$\Psh{\CCx}$};
    \node (PrR) [below = of PrC] {$\Psh{\RRx}$};
    \path[{right hook}->] (C) edge node [above] {$\Yo$} (PrC);
    \path[->] (PrC) edge node [right] {$j^*$} (PrR);
    \path[->,densely dotted] (C) edge node [below,sloped] {$\Nv$} (PrR);
  \end{tikzpicture}
\end{equation}

$\Nv$ is left exact, because it is a composite of left exact functors; it's
worth noting that $\Nv$ is \emph{not} a Cartesian closed functor, because the
change of base $j^*$ does not preserve the exponential.

\section{The gluing construction}

\para In order to specify normalization for $\CCx$, we will need a language much stronger than $\CCx$; the

\section{Notions of neutral and normal forms}

\para Rather than immediately constructing an ad hoc characterization of normal
and neutral forms, we will first specify exactly what is required of these
notions for the main theorem, and then show that such a notion exists. In
subsequent sections, we will then work abstractly with an arbitrary notion of
normal form.

\para Let $\Mor|{right hook}->|[d]{\DDx}{\CCx}$ be the least wide subcategory
of $\CCx$, i.e.\ the free category on the underlying set of objects of $\CCx$.

\para A \emph{notion of normal form} for Cartesian closed categories is defined
to be a functor $\Mor[\DDx]{?}{?}$


% \para The theory of Cartesian closed categories can be expressed in the
% language of finite limits; therefore, we have a locally finitely presentable
% category $\CCC$ whose objects are Cartesian closed categories equipped with
% choices of product and exponential, and whose arrows are Cartesian closed
% functors which preserve the choice of product and exponential.
%
% \para We have a functor $\Mor{\CAT}{}$
%
\end{document}
