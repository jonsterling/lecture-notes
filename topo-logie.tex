\documentclass{jon-notes}

\DeclareMathOperator\colim{colim}
\NewDocumentCommand\Set{}{\mathbf{Set}}
\NewDocumentCommand\CC{}{\mathbb{C}}
\NewDocumentCommand\Psh{m}{\mathbf{P}(#1)}
\NewDocumentCommand\Sh{m}{\mathbf{Sh}(#1)}
\NewDocumentCommand\BigPsh{m}{\overline{\mathbf{P}}(#1)}
\NewDocumentCommand\Sp{}{\mathcal{S}}
\NewDocumentCommand\Hom{mm}{[#1,#2]}
\NewDocumentCommand\OpCat{m}{#1^{\mathrm{op}}}
\NewDocumentCommand\cat{}{\mathbf{cat}}
\NewDocumentCommand\Cat{}{\mathbf{Cat}}
\NewDocumentCommand\CAT{}{\mathbf{CAT}}
\NewDocumentCommand\CAAT{}{\mathbb{CAT}}
\NewDocumentCommand\Pres{}{\mathbf{Pres}}
\NewDocumentCommand\Loc{mm}{\mathbf{L}(#1,#2)}
\NewDocumentCommand\UU{}{\mathbb{U}}
\NewDocumentCommand\U{}{\mathrm{U}}

\DeclareFontFamily{U}{min}{}
\DeclareFontShape{U}{min}{m}{n}{<-> dmjhira}{}
\NewDocumentCommand{\hirayo}{}{\text{\usefont{U}{min}{m}{n}\symbol{'110}}}
\NewDocumentCommand\Yo{s g}{%
  \mkern-4mu\hirayo%
  \IfValueT{#2}{%
    \IfBooleanTF{#1}{
      \parens{#2}
    }{#2}
  }%
}


\addbibresource{references/refs.bib}

\title{The theory of logoi (M.\ Anel)}
\author{Notes by Jon Sterling}
\date{August 9, 2019}

\begin{document}
\maketitle

\begin{abstract}

  These notes were prepared from a course given by Mathieu Anel at HoTT 2019
  Summer School~\citep{anel:2019:hott}; all mistakes are my own. See
  \citet{anel-joyal:2019,anel:2019:hottest} for more detailed material on this
  topic.

\end{abstract}
\bigskip

\para
In these notes, everything is implicitly $\infty$-categorical; 1-categories
are explicitly marked.

\section{Course 2: the theory of logoi}

\para
A topos is a space defined by a \emph{logos of sheaves}; a logos is an algebraic structure on the category of sheaves on a space.


\para The dichotomy between sets and classes is not enough to handle category
theory.  An inaccessible cardinal is a cardinal $\alpha$ such that
$\Set_{<\alpha}$ is stable under dependent product and sum. We will use the
terms atomic (subsingleton), finite, small, normal, large for several strictly
increasing scales of inaccessible cardinal.  The category of small categories
is a normal category.

We will write $\Sp$ for the normal $\infty$-category of small spaces, and
$\overline{\Sp}$ for the large $\infty$-category of normal spaces.


\subsection{Free cocompletion}

\para Let a category $C$ be called \emph{cocomplete} if for any small category $K$, the
functor of constant diagrams $\Mor{C}{C^K}$ has a left adjoint.
%
A functor $\Mor{C}{D}$ between cocomplete functors should be
\emph{cocontinuous}; this gives a category of cocomplete categories.
%
The forgetful functor from cocomplete categories to complete categories has a
left adjoint (Lurie). This is the \emph{free cocompletion}.

\para If $C$ is small, the $\Psh{C} = \Hom{\OpCat{C}}{\Sp}$ is a category. If
$C$ is normal, then we take presheaves $\BigPsh{C}$ into $\overline{\Sp}$ and
take the smallest subcategory of this large presheaf category which is closed
under small colimits. This can be obtained from a transfinite iteration.

  An object $X:C$ is \emph{small} if for all filtered
  $\Mor[Y]{K}{C}$, we have $\Hom{X}{\colim_k Y_k} = \colim_k\Hom{X}{Y_k}$

\para\label{para:presentability} $C$ is \emph{presentable} if there exists a small diagram $\Mor{K}{C}$
and a small diagram $W\to\Psh{K}^{\to}$, such that $\Loc{\Psh{K}}{W}\simeq C$;
in other words, $C$ is the localization of a presheaf category at some class of
maps.
%
 A cocomplete category is small iff it is presentable; this statement must be
 verified within a specific \emph{model} of $\infty$-categories, such as
 quasicategories.



\subsection{Logos and descent}

Assume that $C$ is a category with finite limits (a \emph{lex category}).

\para A \emph{family} in $C$ parameterized by $B:C$ is a map $\Mor[p]{E}{B}$. A
morphism of families over the same base is a commuting triangle. The category
of such things is the slice $C/B$.
%
The reindexing of a family $\Mor[p]{E}{B}$ along an arbitrary map $\Mor{B'}{B}$
is simply the fiber product. From this we construct the functor
$\Mor[\UU]{\OpCat{C}}{\CAT}$ which takes an object to its slice, and a morphism
to the fiber product. We call $\UU$ the \emph{universe} of $\CC$; this functor
gives the fibers of the codomain fibration. In 1-category theory, $\UU$ would
be a pseudofunctor into a bicategory.

\para
We define $\U$ as the groupoid core of $\UU$: it has the same objects as $\UU$
but only the invertible arrows. This universe in the sense of type theory; but
neither is actually an object of $C$; they are structures \emph{over} $C$.
%
Suppose we have a small diagram $\Mor[X]{I}{C}$ and the colimit $X_\infty =
\colim_i X_i$; what if we want to describe a family over the colimit, i.e.\
$\Mor{E}{X_\infty}$? This is the topic of \emph{descent}.

\para We obtain a ``Yoneda'' embedding $\Mor[\Yo]{C}{\Hom{\OpCat{C}}{\CAT}}$,
factoring through actual Yoneda embedding and the inclusion of groupoids into
categories. The Yoneda lemma says that morphisms $\Mor[\phi]{\Yo{B}}{\UU}$ are
the ``same'' as families $\overline{\phi}\in \UU(B)=C/B$. $\UU$ classifies
families and all morphisms (not only isomorphisms as $\U$, or only monos as
$\Omega$).

\para
Coming back to descent, we can say that families $\Mor{E}{X_\infty}$ are the
same as maps $\Mor{\Yo{X_\infty}}{\UU}$; by reindexing, we obtain maps
$\Mor{X_k}{\UU}$ for each $k:K$. Going the other direction, can we start from
such component families and get a family over the colimit? At first glance
looks like we might be able to, but note that we are constructing a map
$\Mor{\Yo{X_\infty}}{\UU}$ and the representable object on the left only acts
as a colimit with respect to representable objects on the right: but $\UU$ is
not representable.

We want a unique lifting condition: if we define a family out of the
$\Psh{C}$-colimit, can we get a unique map out of the Yoneda embedding of the
$C$-colimit? In other words, can $\UU$ distinguish between the colimit and the
Yoneda embedding of the colimit? This is descent. Phrasing this more precisely, we want
$
  \Mor{\Hom{\Yo{\colim X}}{\UU}}{\Hom{\colim \Yo{X}}{\UU}}
$
to be an equivalence of categories.

We can pull the colimit out of the hom as a limit, and then this equivalence is
calculated to the more familiar descent condition $\lim_i\Hom{\Yo{X_i}}{\UU}
\simeq \Hom{\Yo{\colim_i{X_i}}}{\UU}$, i.e.\ $\lim_i{(C/X_i)} \simeq
C/{\colim_i X_i}$. There is an obvious adjunction here, given by reindexing and
colimit. \emph{Descent says that this adjunction is an equivalence.}


\subsubsection{Two conditions for descent}
\para
Start with $\Mor{E}{X_\infty}$ and then pull it back to get the fiber product
$E\times_{X_\infty} X_j$ and take the colimit; that $E=\colim_j
(E\times_{X_\infty} X_j)$ is called the \emph{universality of colimits}. This is
satisfied by any cocomplete, locally Cartesian closed category.

\para\label{para:effectivity}
The hard direction is when you start from a family of diagrams $\Mor{E_i}{X_i}$,
take the colimit, and then pull it back. We want $(\colim_i
E_i)\times_{X_\infty} E_j = E_j$ for each $j$; Mathieu calls this condition
\emph{effectivity}, by analogy with the effectivity of equivalence relations in
classical topos theory.

\para
non-trivial $n$-categories do not have descent; $\infty$-categories can have descent.

\section{Course 3: features of logoi}

We begin by reviewing some counterexamples to descent.
% \subsection{Counterexamples to descent}

\para In $\Set$ the colimits are universal, but effectivity of colimits fails.
He gives the counterexample with $S^0=2$:
\[
  \begin{tikzpicture}[on grid,node distance=2cm]
    \node (S01) {$S^0$};
    \node (S0S0) [right = of S01] {$S^0\times S^0$};
    \node (S02) [right = of S0S0] {$S^0$};
    \node (11) [below = of S01] {$1$};
    \node (S03) [below = of S0S0] {$S^0$};
    \node (12) [below = of S02] {$1$};
    \path[->] (S01) edge (11);
    \path[->] (S02) edge (12);
    \path[->] (S03) edge (11);
    \path[->] (S03) edge (12);
    \path[->] (S0S0) edge (S01);
    \path[->] (S0S0) edge (S02);
    \path[->] (S0S0) edge node [upright desc] {$m$} (S03);
  \end{tikzpicture}
\]

Note that the colimit of the spans upstairs and downstairs is just $1$, but if
you take the pullbacks described in \ref{para:effectivity}, one does not obtain
the desired fibers. For instance:

\[
  \begin{tikzpicture}[on grid,node distance=2cm]
    \node [pullback] (1) {$1\not= S^0$};
    \node (1') [below = of 1] {$1$};
    \node (colim) [right = of 1] {$Y_\infty=1$};
    \node (colim') [right = of 1'] {$X_\infty=1$};
    \path[->] (1) edge (1');
    \path[->] (1) edge (colim);
    \path[->] (1') edge (colim');
    \path[->] (colim) edge (colim');
  \end{tikzpicture}
\]

\para In 1-$\mathbf{Gpd}$, you get an analogous counterexample from $S^1$.


\NewDocumentCommand\LOGOS{}{\mathbf{LOGOS}}
\NewDocumentCommand\Logos{}{\mathbf{Logos}}

\para A cocomplete lex category is called a \emph{logos}; these are arranged
into a category whose morphisms are cocontinuous lex functors. We will write
$\LOGOS$ for the category of logoi. There are forgetful functors to cocomplete
cateogires, lex categories, and categories.

\para The \emph{presentable logoi} are the logoi whose underlying cocomplete
categories are presentable in the sense of \ref{para:presentability}; the
category of such presentable logoi is written $\Logos$.


\para An object of $\Logos$ is an $\infty$-topos in the sense of \citet{lurie:2009}.

\para A general logos need not be locally Cartesian closed, and morphisms $f^*$
between logoi need not have right adjoints. This is the price to pay for the
category $\LOGOS$ to be nice.

\para (Anel-Lejay) The following are equivalent:
\begin{enumerate}
  \item $C$ is a logos
  \item The canonical cocontinuous functor $\Mor{\Psh{C}}{C}$ is left exact. (Analogous to pre-topoi.)
  \item The left Kan extension of every lex diagram $\Mor{D}{C}$ along $\Mor{D}{\Psh{D}}$ is lex.
  \item Giraud axioms: like for $\infty$-topoi, removing presentability.
\end{enumerate}

\para There is an analogy with commutative algebra; presentable logoi are like
finitely generated commutative rings. Grothendieck made general commutative
rings the main object of study rather than the finitely generated ones, and
Mathieu advocates that we do the same in topos theory; the category of general
logoi is nicer than the category of presentable logoi.

\subsection{Other examples of logoi}

\para The free logos on a category: one takes $C$ to $\Sp[C] =
\Psh{C^{\mathit{lex}}}$. In particular, the free logos on one generator $S[X]$
is $\Hom{\mathbf{Fin}}{\Sp}$.

\para If $T$ is a Lawvere theory, then it is a category with finite products.
Let the \emph{enveloping logos} $\Sp\langle T\rangle =
\Psh{T^{\mathit{lex}/\times}}$. This logos classifies models of the theory $T$.

\para $\infty$-sheaves on a locale $X$; we can construct a 1-logos $\Sh{X}$ out
of functors $\Hom{\OpCat{\mathcal{O}(X)}}{\Set}$ which satisfy the sheaf
condition for the canonical topology. This generalizes to $\infty$-logoi in a
straightforward way, where the sheaf condition is strengthened to consider not
just a couple levels of overlap, but infinitely many. This $\infty$-sheaf
condition restricts to the finite-dimensional versions when the objects
involved are truncated.


\nocite{lurie:2009,hottbook}
\printbibliography

\end{document}
for a group $G$.
